Throughout this thesis, we have been probing distinct methods to measure the complexity of distinct objects. We have proved that in all the cases the best measurement of complexity is obtained by trying to approximate Kolmogorov complexity. Thus, this implementation is recommended and can be safely used in any application for which there is a necessity to measure complexity. Nonetheless, we have shown that for certain mathematical objects this implementation must be used with care. If the mathematical object we are trying to measure its complexity has distinct isomorphic representations, then the true complexity of the object is the minimum value obtained with one of these isomorphic representations. Besides, for graphs or any mathematical object which could be represented as a matrix, we have seen how it is better to transform this representation to a 1-dimensional representation to get better measurements of complexity. These results are important since they will make easier to perform future experiments about complexity.\\

Moreover, we proposed a novel application to Kolmogorov complexity by measuring the complexity of Boolean networks. To do so, we needed to propose a representation for Boolean Networks which were representative of its behavior and could be used to measure its complexity. This representation was evaluated by using it to prove some hypothesis of correlation among the complexity of a Boolean Network and the complexity of its constituents. The results agreed with which was expected which means our approach works correctly and the representation used correctly captures the essential features which characterize the complexity of a Boolean Network. Although, future experiments with larger ensembles of Boolean Networks will be needed to be able to generalize the results and confirm these hypotheses for Boolean Networks of any size. Even so, the results of our proposal to measure the complexity of Boolean Networks are promising and they could be used in the study of real-world systems which can be modeled by means of Random Boolean Networks. For instance, we could use the measurements of the complexities of the same genetic regulatory network for different species (modeled as Boolean Networks) to try to establish phylogenetic relationships. This is because we could expect the complexity of a genetic regulatory network to be lower for species who evolved earlier than others. Of course, our methods and observations obtained by measuring the complexity of graphs, digraphs, and binary sequences could be used to study real-world systems as well.\\

Our experiments also showed that the library we used to measure K-complexity works well, though it needs a lot of time to compute the complexity for long sequences. This problem makes it not suitable to study the complexity of objects whose representation is large. Therefore, given the enormous amount of possible applications where it is needed to compute Kolmogorov complexity. We decided to propose a novel method which could be used in situations where we need a faster implementation to measure complexity. Our implementation called BDMNN works faster and needs very low computational power. The only cost to pay is a little loss in accuracy, however, we believe that this implementation can be easily enhanced to reduce the error of approximating K-complexity by training more Neural Networks with larger data sets and larger training times. This novel approach works quite similar to the original BDM and in the worst cases behaves just like Shannon entropy, so it can be used immediately for any application. Future experiments could help to determine if a regression with any other machine learning technique works better. Certainly, it could be used in future experiments to generalize the results of our hypothesis of correlation to Boolean Networks with larger topologies.

%Finally, it has to be remarked that the original BDM implementation was created by considering an alphabet with only two elements ($0$ and $1$), so we can not expect well results when measuring the complexity of objects whose representation needs more alphabet elements. If in the future


%if there exists a genetic regulatory network which is trying to be modeled as a Boolean Network and the topology of this network is already known but its updating functions are not. We could find the set of updating functions by using the complexity of its state space (the complexity of the Boolean Network) and the complexity of the topology to discard some sets whose complexities does not fit with a linear correlation. In this sense, it could be helpful in a future experiment, to evaluate the 

%Nevertheless, basados en alfabeto 0 y 1
