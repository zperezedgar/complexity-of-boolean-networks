\pagenumbering{arabic} % fin num romana


The Boolean Networks are a type of discrete dynamical system which can be used to model a lot of real-world networks. They have been used especially to model genetic regulatory networks by means of the Random Boolean Network model introduced by Kauffman in 1969. These models have proven to be worth to model not only genetic networks but also many other types of systems. It is expected that the ability of a Boolean Network to model any complex system, like a living system, must be intrinsically related to the complexity of any of its constituents (its topology and its set of updating functions). If its constituents are not complex enough, then the complexity of the Boolean Network also will not be enough. In other words, there should exist a correlation among the complexity of a Boolean Network and the complexity of its topology (or its set of updating functions).\\

In this thesis, we will try to prove the previous hypothesis of correlation. To do so, in Chapter \ref{graph_theory_chapter}, we will begin by giving a little review to graph theory since this is the mathematical tool which allows the description of the topology of any type of network. We will present some basic definitions and the terminology which will be used extensively in the subsequent chapters. A section will be devoted to the study of directed digraphs since they are used as the topology for Boolean Networks. Besides, some distributions of random graphs which will be used later will be studied.\\

Afterward, in Chapter \ref{boolean_chapter}, the model of a Boolean Network will be studied with an emphasis on the Random Boolean Network model proposed by Kauffman. We will study how these models are built from randomly choosing a topology and a set of updating functions. We will talk about the features of these constituents and the possible updating schemes which can be chosen for the dynamics of the network. Finally, we will review some real-world applications of this model and the software which can be used to study them.\\

In Chapter \ref{algorithmic_complexity_chapter}, a review of algorithmic complexity will be given. This chapter is the cornerstone of this thesis. Here, the definition of Kolmogorov complexity will be given. Then, we will study some approaches which have been used to approximate this complexity. Thereupon, we will review some theorems which have been used to create an implementation which can be used to estimate K-complexity. Following, some details about this implementation called Block Decomposition Method (BDM) will be given. This chapter finishes by giving some applications of Kolmogorov complexity.\\

The second part of this thesis starts with Chapter \ref{kolmo_rbn_chapter}. In this chapter, we will present the experiments performed to prove our hypothesis about the existence of a correlation between the complexity of a Boolean Network and the complexity of its constituents. Firstly, we establish the methods used to measure Kolmogorov complexity. This is done by performing some previous experiments with random sequences of bits and random graphs. Then, we move on to measure the complexity of random digraphs, i.e., the complexity of the topologies of a Boolean Network. We will check distinct types of representations and its effect in the complexity which is obtained. After, we will present a method to measure the complexity of a set of Boolean functions, the other constituent of a Boolean Network. Once we have learned how to measure the complexity of the individual constituents of a Boolean Network, we will continue by proposing a way to represent a Boolean Network as a sequence of bits. This representation is then used to measure the complexity of Boolean Networks and to test our hypothesis of correlation by performing some experiments with Random Boolean Networks.\\

The results of these experiments showed that the representations used were able to capture the features which give complexity to the mathematical objects which were studied. The hypothesis of correlation was proven to be correct for Random Boolean Networks with small topologies given an ensemble of random networks sufficiently large.\\

Unfortunately, these results could not be generalized to Boolean Networks with larger topologies because of the enormous computational time demanded by the implementation used to approximate Kolmogorov complexity. Inspired by this trouble, in Chapter \ref{NN_proposal_chapter}, we propose a novel method to approximate Kolmogorov complexity by using Neural Networks. These Neural Networks were used to perform a regression which approximates Kolmogorov complexity. The Neural Networks were trained by using random sequences of bits which complexity was computed using the original BDM implementation. We have called this method the Block Decomposition Method with Neural Networks (BDMNN). Therefore, in this chapter, we will give a brief introduction to Machine Learning and Neural Networks. Then, the methodology and the details to create this implementation are presented. Finally, this implementation is evaluated by performing some experiments to compare its results with the results of the original implementation. The results showed that our method is faster and the only cost which must be paid is a little reduction in the accuracy of the approximation, nevertheless we expect the error can be easily reduced by enhancing the method through some little modifications which we talk about in the last part of this chapter.\\

The codes and algorithms used to perform the experiments of this thesis can be consulted in the Appendix section.


%\section{qué voy a hacer y porqué}

%\section{cómo lo voy a hacer}

%\section{estructura de la tesis}

%\section{Resultados}


%en la introducción se ubica al lector en el problema conveciéndolo de la importancia de atenderlo o abordarlo, se presenta la estructutra del trabajo y se citan los agradecimientos a las personas o instituciones que participaron en el mismo. La introducción es la parte en la que se exponen tanto la razón como las finalidades del trabajo de investigación, como las fuentes consultadas, el valor de los datos, el análisis y la comprobación de los mismos, el método empleado y el análisis del esquema de trabajo.